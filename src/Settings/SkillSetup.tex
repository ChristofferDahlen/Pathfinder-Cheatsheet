\NewDocumentCommand{\TextOutline}{O{1pt} O{B} O{0ex} m O{white} O{black} }{%
    \begin{tikzpicture}[baseline=(A.base)]%
        \node[outer sep=0, inner sep=0] (A) {%
            \textpdfrender{%
                TextRenderingMode=FillStrokeClip,%
                LineCapStyle=ProjectingSquare,
                LineJoinStyle=Bevel,%
                LineWidth=#1,%
                StrokeColor=#6,%
                FillColor=#5%
            }{#4}%
        };
        \node[outer sep=0, inner sep=0, text=#5, anchor=north west] at (A.north west) {%
            #4%
        };%
    \end{tikzpicture}%
}

\NewDocumentCommand{\LVL}{m}{{\rmfamily\scalefont{1.1}\IBoxX{\MSC{Lv}L \Numb{#1}}}}%

\newcommand{\TimeBox}[1]{%
    \tcbox[on line,
        box align=base,
        arc=0.0ex,
        halign=center,
        valign=center,
        colframe=TimeColorBorder,
        colback=TimeColor,
        right skip=0ex,
        left skip=0ex,
        size=fbox,
        top =0.2ex,
        bottom=0.1ex,
        left=0.3ex,
        right=0.3ex,
        boxsep=0.1ex]{\textcolor{TimeColorText}{\textbf{#1}}}%
}

\NewDocumentCommand{\Time}{O{} m O{} m O{}}{{\scriptsize\scalefont{0.6}#3%
\IfNotBlank{#1}{#1\,}%
\IfNotBlank{#2}{{\Numb{#2}}\,}%
#4%
\IfNotBlank{#5}{#5}%
}}
%\NewDocumentCommand{\Time}{O{} m m O{}}{{\scriptsize\scalefont{0.5}%
%    \IfNotBlank{#1}{#1\xspace}%
%    \IfNotBlank{#2}{{\kirsty#2\xspace}}%
%    \IfNotBlank{#3}{#3\xspace}#4}}
\NewDocumentCommand{\TimeMin}{m O{} O{}}{\TimeBox{\Time{#1}[#3]{\MSC{Min}}[#2]}}
\NewDocumentCommand{\TimeHour}{m O{} O{}}{\TimeBox{\Time{#1}{H}[#2]}}
\NewDocumentCommand{\TimeDay}{m O{} O{}}{\TimeBox{\Time{#1}[#3]{\MSC{Day}}[#2]}}
\NewDocumentCommand{\TimeDays}{m O{} O{}}{\TimeBox{\Time{#1}[#3]{\MSC{Days}}[#2]}}
\NewDocumentCommand{\TimeWeek}{m O{} O{}}{\TimeBox{\Time{#1}[#3]{\MSC{Week}}[#2]}}
\NewDocumentCommand{\TimeWeeks}{m O{} O{}}{\TimeBox{\Time{#1}[#3]{\MSC{Weeks}}[#2]}}
\NewDocumentCommand{\TimeDaily}{O{}}{\TimeBox{\Time{\MSC{#1Daily}}}}

\NewDocumentCommand{\TimeHex}{O{} m}{\TimeBox{\Time[#1]{#2}{\MSC{Hex Act.}}}}


\NewDocumentCommand{\tagBox}{m O{} m}{%
        \tcbox[%
            capture=hbox,
            colframe=tagFieldColor,
            colback=#3,
            halign=center,
            valign=center,
            size=fbox,
            on line,
            right skip=0.1ex,
            left skip=0.1ex,
        %before skip=0pt,
        %after skip=0pt,
            sharp corners,
            boxrule=0.1ex,
            boxsep=0.1ex,
            right=0.1ex,
            left=0.1ex,
            top=0.1ex,
            bottom=0.0ex,
            enlarge top initially by=-0.2ex,
            enlarge bottom finally by=-0.1ex,
            box align=base,
        ]{\scalefont{0.8}\sffamily{\TextOutline[0.5pt][M]{\textbf{\MSC{#1}#2}}}}%
}%



\NewDocumentCommand{\SkillTextOutline}{m O{}}{\TextOutline[1pt][#2]{#1}}

\NewDocumentCommand{\Feat}{}{%
    \SpecialBox[color text=FeatColorText,
        color box border=FeatColorBorder,
        color box background=FeatColor
    ]{Feat}%
}

\NewDocumentCommand{\Rare}{O{} O{}}{%
    \SpecialBox[color text=RareColorText,
        color box border=RareColorBorder,
        color box background=RareColor
    ]{Rare}%
}

\NewDocumentCommand{\Unique}{O{} O{}}{%
    \SpecialBox[color text=UniqueColorText,
        color box border=UniqueColorBorder,
        color box background=UniqueColor
    ]{Unique}%
}

\NewDocumentCommand{\Acrobatics}{O{} O{} O{}}{\SkillBoxX{skill=Acrobatics, #1}[#2][#3]}
\NewDocumentCommand{\Deception}{O{} O{} O{}}{\SkillBoxX{skill=Deception, #1}[#2][#3]}
\NewDocumentCommand{\Crafting}{O{} O{} O{}}{\SkillBoxX{skill=Crafting, #1}[#2][#3]}
\NewDocumentCommand{\Perception}{O{} O{} O{}}{\SkillBoxX{skill=Perception, #1}[#2][#3]}
\NewDocumentCommand{\Stealth}{O{} O{} O{}}{\SkillBoxX{skill=Stealth, #1}[#2][#3]}
\NewDocumentCommand{\Survival}{O{} O{} O{}}{\SkillBoxX{skill=Survival, #1}[#2][#3]}
\NewDocumentCommand{\Athletics}{O{} O{} O{}}{\SkillBoxX{skill=Athletics, #1}[#2][#3]}
\NewDocumentCommand{\Thievery}{O{} O{} O{}}{\SkillBoxX{skill=Thievery, #1}[#2][#3]}
\NewDocumentCommand{\Diplomacy}{O{} O{} O{}}{\SkillBoxX{skill=Diplomacy, #1}[#2][#3]}
\NewDocumentCommand{\Performance}{O{} O{} O{}}{\SkillBoxX{skill=Performance, #1}[#2][#3]}
\NewDocumentCommand{\Nature}{O{} O{} O{}}{\SkillBoxX{skill=Nature, #1}[#2][#3]}
\NewDocumentCommand{\Society}{O{} O{} O{}}{\SkillBoxX{skill=Society, #1}[#2][#3]}
\NewDocumentCommand{\Arcana}{O{} O{} O{}}{\SkillBoxX{skill=Arcana, #1}[#2][#3]}
\NewDocumentCommand{\Piloting}{O{} O{} O{}}{\SkillBoxX{skill=Piloting, #1}[#2][#3]}
\NewDocumentCommand{\Medicine}{O{} O{} O{}}{\SkillBoxX{skill=Medicine, #1}[#2][#3]}
\NewDocumentCommand{\Flat}{O{} O{} O{}}{\SkillBoxX{skill=Flat, #1}[#2][#3]}
\NewDocumentCommand{\Fortitude}{O{} O{} O{}}{\SkillBoxX{skill=Fortitude, #1}[#2][#3]}
\NewDocumentCommand{\Will}{O{} O{} O{}}{\SkillBoxX{skill=Will, #1}[#2][#3]}
\NewDocumentCommand{\Reflex}{O{} O{} O{}}{\SkillBoxX{skill=Reflex, #1}[#2][#3]}
\NewDocumentCommand{\MagicalSkill}{O{} O{} O{}}{\SkillBoxX{skill=Magical Skill, #1}[#2][#3]}
\NewDocumentCommand{\VariousKnowledge}{O{} O{} O{}}{\SkillBoxX{skill=Various Knowledge, #1}[#2][#3]}
\NewDocumentCommand{\AthleticsFortitude}{O{} O{} O{}}{\SkillBoxX{skill=Athletics, defense=Fortitude, #1}[#2][#3]}
\NewDocumentCommand{\AcrobaticsReflex}{O{} O{} O{}}{\SkillBoxX{skill=Acrobatics, defense=Reflex, #1}[#2][#3]}
\NewDocumentCommand{\DeceptionWill}{O{} O{} O{}}{\SkillBoxX{skill=Deception, defense=Will, #1}[#2][#3]}
\NewDocumentCommand{\InitimidationWill}{O{} O{} O{}}{\SkillBoxX{skill=Intimidation, defense=Will, #1}[#2][#3]}
\NewDocumentCommand{\DiplomacyWill}{O{} O{} O{}}{\SkillBoxX{skill=Diplomacy, defense=Will, #1}[#2][#3]}
\NewDocumentCommand{\NatureWill}{O{} O{} O{}}{\SkillBoxX{skill=Nature, defense=Will, #1}[#2][#3]}
\NewDocumentCommand{\PerceptionStealth}{O{} O{} O{}}{\SkillBoxX{skill=Perception, defense=Stealth, #1}[#2][#3]}
\NewDocumentCommand{\AthleticsReflex}{O{} O{} O{}}{\SkillBoxX{skill=Athletics, defense=Reflex, #1}[#2][#3]}
\NewDocumentCommand{\DeceptionPerception}{O{} O{} O{}}{\SkillBoxX{skill=Deception, defense=Perception, #1}[#2][#3]}


\NewDocumentCommand{\DC}{O{}}{%
    \ifthenelse{\equal{#1}{}}{\Typetv{DC}}{\DCBoxX{val=#1}}%
}
